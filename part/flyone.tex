\chapter{基于波数扫描的傅立叶变换白光干涉测量术}

\section{光纤白光干涉光谱信号在波长域上的啁啾特性}
在波长域中沿着扫描波长分布的白光干涉光谱信号通常可以表示为
\begin{equation}
g(\lambda)=a(\lambda)+b(\lambda)cos(2\pi f_{\lambda}+\pi)
\end{equation}


其中$\lambda$表示扫描波长,$f_{\lambda}$表示波长域中干涉光谱信号的频率,$f_{\lambda}=2d^{2}/\lambda^{2}$,d为光纤法珀传感器的腔长,2d即为光纤法珀传感器的光程差。$a(\lambda)$是由宽带光源引入的背景光信号,$b(\lambda)$是白光干涉条纹的对比度变化,与光纤法珀腔中的光纤端面的反射有关。相位项中的常数$\pi$由法珀腔中的第二个端面的反射引入。波长域白光干涉光谱的周期可以表示为
\begin{equation}
T_{\lambda}=\lambda^{2}/(2d)
\end{equation}


当光纤法珀传感器的腔长 d 一定时,干涉光谱的周期随扫描波长非线性地变化。
\section{基于波数扫描的傅立叶变换白光干涉测量术 }
\subsection{波数域傅立叶变换白光干涉测量原理 }
由于白光干涉光谱的相位与扫描波长之间是反比的关系,干涉光谱信号的频率与扫描波长之间存在非线性关系,因此可以采用波数作为变量来重新描述白光干涉光谱。白光干涉光谱在波数域可以表示为 
\begin{equation}
g(k)=a(k)+b(k)cos(2\pi f_{0}k+\pi)
\label{flyone1}
\end{equation}


其中 k 表示波数,波数 k 与波长 $\lambda$之间是反比关系,$k=1/\lambda_{0}$。$f_0$是波数域中白光干涉光谱信号的频率,$f_{0}=2d$。在波数域上白光干涉光谱的周期为
\begin{equation}
T=1/(2d)
\end{equation}


可以看出,在波数域中白光干涉光谱的周期不随波数的变化而改变,只与光纤法珀传感器的光程差有关。当光纤法珀传感器的光程差一定时,可以认为白光干涉光谱的周期 T 为一个恒定值。因此沿波数分布的白光干涉光谱信号是一个等周期信号,光谱信号中不存在啁啾。在对白光干涉光谱信号进行傅立叶变换时,不会因为啁啾而引起傅立叶频谱展宽。根据欧拉公式,方程\ref{21}可以写为如下形式 